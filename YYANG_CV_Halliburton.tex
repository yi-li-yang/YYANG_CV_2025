%%%%%%%%%%%%%%%%%%%%%%%%%%%%%%%%%%%%%%%%%
% Medium Length Professional CV
% LaTeX Template
% Version 2.0 (8/5/13)
%
% This template has been downloaded from:
% http://www.LaTeXTemplates.com
%
% Original author:
% Rishi Shah 
%
% Important note:
% This template requires the resume.cls file to be in the same directory as the
% .tex file. The resume.cls file provides the resume style used for structuring the
% document.
%
%%%%%%%%%%%%%%%%%%%%%%%%%%%%%%%%%%%%%%%%%

%----------------------------------------------------------------------------------------
%	PACKAGES AND OTHER DOCUMENT CONFIGURATIONS
%----------------------------------------------------------------------------------------
\renewcommand{\baselinestretch}{0.70}
\documentclass{resume} % Use the custom resume.cls style
\newenvironment{indentpar}[1]%
  {\begin{list}{}%
          {\setlength{\leftmargin}{#1}}%
          \item[]%
  }
  {\end{list}}

\usepackage[T1]{fontenc}
\usepackage[utf8]{inputenc}
\usepackage{mathptmx}

\usepackage[left=0.5in,top=0.35in,right=0.5in,bottom=0.4in]{geometry} % Document margins
\newcommand{\tab}[1]{\hspace{.25\textwidth}\rlap{#1}}
\newcommand{\itab}[1]{\hspace{0em}\rlap{#1}}
\name{YILI YANG} % Your name
\address{07490334233  \textbar  yyl.eli@gmail.com;  \textbar
 Room 400 the Grant Institute, James Hutton Road, Edinburgh EH9 3FE} % Your phone number and email

\begin{document}
%=================================================================================================================================================
\begin{rSection}{Education}
{\bf University of Edinburgh, UK} 
\begin{indentpar}{0.5cm}

\textbf{PhD Candidate in Geology} \hfill {Expected 01/2021}
\begin{indentpar}{0.5cm}
Project title: \textit{Multiphase Fluid Flow and Trapping in Porous Carbonates: A synchrotron X-ray CT and pore-scale modelling investigation. (Fully funded by Petrobras and Shell)}
\end{indentpar}
\textbf{MSc(Research) in Geology, Distinction} \hfill {August 2015 - August 2016} 
\begin{indentpar}{0.5cm}
Project title: \textit {Processing and analysis of time-resolved X-ray μCT data of experimental dolomitisation}
\end{indentpar}
\textbf{BSc Geology, 2.1} (2+2 double-degree programme with CDUT) \hfill {Sep 2011 - May 2015} 
\end{indentpar}
\end{rSection}
%=================================================================================================================================================
\begin{rSection}{Technical Strengths}

\begin{indentpar}{0.5cm}
\textbf{Programming Languages} Python (5-year experience) with good knowledge of science and data related libraries such as Scipy, Pandas, PyTorch, Sk-Learn and Tomopy etc. Experience in Command line, SQL and LaTeX.

\textbf{Python software/tool development} CT image reconstruction, Image segmentation, Auto calculate REV, Pore-throat measurement, CT image denoise, Image processing workflow

\textbf{Machine learning} Deep learning image segmentation using convolutional neural networks.

\textbf{Professional Software/Tools} Avizo, ImageJ, Octopus, Dragonfly, ParaView, Tableau, Git

\end{indentpar}
\end{rSection}

%=================================================================================================================================================
\begin{rSection}{Publications and Conferences}
\textbf{Publications}
\begin{indentpar}{0.5cm}
\begin{itemize}
    \item \textbf{Yang et.al.}, 2020, An implementation of a convolutional neural network for fast segmentation of 4D microtomography volumes from fluid flow experiments in porous media, submitted to Water Resources Research
    \item Maes and \textbf{Yang}, 2019, Numerical investigation of a Roof snap-off observed in an Indiana limestone core flood experiment, submitted to Arizona Department of Water Resources
    \item Sina Marti et al., 2020, 
    Time-resolved grain-scale 3D imaging of hydrofracturing in halite layers induced by gypsum dehydration and pore fluid pressure buildup, Earth and Planetary Science Letters
    \item Sina Marti et al., 2019, Chemical-mechanical-hydraulic coupling in deforming, dehydrating halite-gypsum rocks - implications for basal detachments in thin-skinned tectonics, in prep.
\end{itemize}
\end{indentpar}

\textbf{Conferences}
\begin{indentpar}{0.5cm}
\textbf{Interpore}, Valencia, Spain \hfill {2019}
\begin{indentpar}{0.5cm}
\textit {Speak: "Fluid displacement and trapping during two-phase steady-state flow in complex carbonate imaged by synchrotron x-ray microtomography"}
\end{indentpar}

\textbf{AGU Fall}, Washinton DC, U.S. \hfill {2018}
\begin{indentpar}{0.5cm}
\textit {Speak: "Fluid Connectivity Evolution during Drainage-Imbibition Cycle in Porous Carbonate Rock"}
\end{indentpar}

\textbf{ICCR Conference} ICCR Annual Conference with Shell and Petrobras, RJ, Brazil \hfill {2018}
\begin{indentpar}{0.5cm}
\textit {Speak: "SatuTrack II - Experimental Multi-phase Fluid Displacement in Presalt Reservoir Rocks"}

\textit {Workshop: "Image processing and segmentation for X-ray microtomography"} UFRJ, Brazil
\end{indentpar}

\textbf{ICCR Conference} ICCR Annual Conference with Shell and Petrobras, RJ, Brazil \hfill {2017}
\begin{indentpar}{0.5cm}
\textit {Speak: "SatuTrack II - Saturation tracking and identification of residual oil distributions"}
\end{indentpar}

\textbf{Postgraduate Research Conference} The University of Edinburgh \hfill {2017} 
\begin{indentpar}{0.5cm}
\textit {Speak: "Multiphase Fluid Flow and Trapping in Porous Carbonates"} (Presentation Prize Winner)
\end{indentpar}
\end{indentpar}
\end{rSection}
%=================================================================================================================================================

\newpage
\begin{rSection}{ACADEMIC EXPERIENCES}
\textbf{School of Geosciences, The University of Edinburgh} 
\begin{indentpar}{0.5cm}
\textit{Multiphase Fluid Flow and Trapping in Porous Carbonates (Thesis study)}
\begin{itemize}
    \item Core-flooding experiments on benchmark carbonate imaged with X-ray synchrotron microtomography
    \item Developed a high-quality CT image processing workflow with parallel computation enabled.
    \item Developed an integrated python-based CT image processing toolkit 
    \item Implemented a machine learning (CNN) approach for CT image segmentation
    \item Identification, visualisation, quantification and analysis of various pore-scale displacement events.
\end{itemize}
\end{indentpar}

\textbf{International Centre for Carbonate Reservoirs (ICCR), Petrobras and Shell} 
\begin{indentpar}{0.5cm}
\textit{An Experimental and modelling study of the Pre-salt hydrocarbon reservoir, Brazil, from pore to reservoir-scale}
\begin{itemize}
    \item Provided experimental data and observations to the modelling groups 
    \item Provided image processing support to other experimental groups
    \item Experimental validation of numerical models on multiphase flow in porous media
\end{itemize}
\end{indentpar}
\textbf{Experiments} 

\begin{indentpar}{0.5cm}
\textit{Gypsum dehydration using synchrotron X-ray imaging, PSICHE, Switzerland} \hfill {2019}

\textit{Core-flooding experiments on reservoir rocks using in-house CT imaging, Edinburgh} \hfill {2018}

\textit{Core-flooding experiments on Indiana limestone using synchrotron X-ray imaging, APS, ANL, U.S.} \hfill {2017}
\end{indentpar}

\textbf{Training} Open Pit Mining for the Extraction of Solid Mineral Resources \hfill {2015}

\textbf{Geological Field works}
\begin{indentpar}{0.5cm}
\textbf{Cyprus} Oceanic crust and ophiolite complex\hfill{2015}

\textbf{Inchnadamph, Scotland} Thrust nappes and tectonic window \hfill{2014}

\textbf{Alicante, Spain} Sedimentary basins and carbonate sediments\hfill{2014}

\textbf{Kinlochleven, Scotland} Rock deformations and structural geology\hfill{2014}

\textbf{Helmsdale, Scotland} Petroleum system of the North Sea\hfill{2013}

\textbf{Majiaoba, China} Foreland Basin and imbricate thrust stacks\hfill{2013}

\textbf{Emeishan, China} Emeishan basalt, P-T boundary and sedimentary units\hfill{2012}

\end{indentpar}
\end{rSection}
%=================================================================================================================================================
\begin{rSection}{Work Experience}
\begin{rSubsection}{Teaching Assistant, University of Edinburgh}{June 2016-2019}{}

\begin{indentpar}{0.5cm}
\textbf{Field Demonstrator} Demonstrated in undergraduate geological field trips in Inchnadamph, Kinlochleven and Cyprus. Teaching field observation and mapping techniques. Teaching geological concepts and principles. Collaborated with other demonstrators to cover logistics and HSE during field trips. Certificated emergency first-aider.

\textbf{Course assistant: Geochemical Data Processing and Analysis} Teaching scientific data analysis using python including data processing, quantified, statistical analysis, and scientific data visualisation. Helping students to use fundamental python packages such as Scikit and Matplotlib. Helping students with course works and tests. Collaborating with other demonstrators to organise workshops.
\end{indentpar}

\end{rSubsection}
\begin{rSubsection}{Intern - Brighter Oil Group, Xi'an, China}{2015 Summer Intern}{}

\begin{indentpar}{0.5cm}
Project: Evaluation of velocity string on enhanced hydrocarbon recovery in Sulige gas field, Inner Mongolia, China. Research and presentation focusing on the application feasibility of velocity string as coiled tubing. Analysing the cost-effectiveness of velocity string in Sulige gas field. Studying and comparing the successful case of velocity string application on enhanced hydrocarbon recovery.
\end{indentpar}

\end{rSubsection}
\end{rSection}
%=================================================================================================================================================
\end{document}
