%%%%%%%%%%%%%%%%%%%%%%%%%%%%%%%%%%%%%%%%%
% Medium Length Professional CV
% LaTeX Template
% Version 2.0 (8/5/13)
%
% This template has been downloaded from:
% http://www.LaTeXTemplates.com
%
% Original author:
% Rishi Shah 
%
% Important note:
% This template requires the resume.cls file to be in the same directory as the
% .tex file. The resume.cls file provides the resume style used for structuring the
% document.
%
%%%%%%%%%%%%%%%%%%%%%%%%%%%%%%%%%%%%%%%%%

%----------------------------------------------------------------------------------------
%	PACKAGES AND OTHER DOCUMENT CONFIGURATIONS
%----------------------------------------------------------------------------------------
\renewcommand{\baselinestretch}{0.70}
\documentclass{resume} % Use the custom resume.cls style
\newenvironment{indentpar}[1]%
  {\begin{list}{}%
          {\setlength{\leftmargin}{#1}}%
          \item[]%
  }
  {\end{list}}

\usepackage[T1]{fontenc}
\usepackage[utf8]{inputenc}
\usepackage{mathptmx}

\usepackage[left=0.5in,top=0.35in,right=0.5in,bottom=0.4in]{geometry} % Document margins
\newcommand{\tab}[1]{\hspace{.25\textwidth}\rlap{#1}}
\newcommand{\itab}[1]{\hspace{0em}\rlap{#1}}
\name{YILI YANG} % Your name
\address{ +44 7490334233\hspace{0.3cm}\textbar\hspace{0.3cm} yyl.eli@gmail.com } % Your phone number and email


\begin{document}
%-------------------------------------------------------------
\begin{rSection}{Work Experience}
{\bf Woodwell Climate Research Center, MA, US (remote)} \hfill {Jan 2022 - } 

    \begin{indentpar} {0.5cm} Data Scientist
    
    Project lead: Detection and segmentation of abrupt thaw slumps in the Arctic using deep learning. Collaborated with parallel projects such as wildfire detection and water body detection. Developed a Python package for satellite imagery processing and Earth Engine data processing.
    
    Workshop Organiser
    
    Organised and lectured workshops for machine learning and deep learning. Contents include introductions to AI and ML, principles and theories, best practices with scientific data, deep learning and deep neural networks, image processing and segmentation, practicals with real data.
    \end{indentpar}
    
    {\bf Faculty.ai, London, UK} \hfill {Oct 2021 - Jan 2022} 
    
    \begin{indentpar}{0.5cm}
    Data Science Fellowship Programme: Intensive hands-on training in data science and commercial skills followed by an industry placement
    \end{indentpar}
\end{rSection}
%===============================================================
\begin{rSection}{Technical Strengths}

\textbf{Machine learning and Deep Learning}
    \begin{indentpar}{0.5cm}
    Experienced in training ML and DL models for semantic segmentation tasks, mainly Unet and its variations and vision Transformers. Involved with projects using random forest models. Experienced with major state-of-the-art techniques such as data augmentation, automatic hyperparameter tuning (Hyperopt) and transfer learning. 
    \end{indentpar}
    
\textbf{Image processing}
    \begin{indentpar}{0.5cm}
    Experienced in processing RGB images, X-ray micro CT images and remote sensing images. Experienced in working with various of data formats including GeoTIFF, HDF5 and raw. Experienced with image processing software ImageJ, Avizo and QGIS.
    \end{indentpar}

\textbf{Tech Stack}
    \begin{indentpar}{0.5cm}
    Proficient Python, basic JavaScript and SQL, Google Cloud Platform, Earth Engine, Colaboratory, Jupyter Notebook, VS Code, Git, Jira software management, Tableau. Markdown and \LaTeX.
    \end{indentpar}

    \begin{indentpar}{0.5cm}
    Python libraries: Numpy, Scikit-image, OpenCV, Scikit-learn, Pandas. PyTorch and Tensorflow. Matplotlib and Seaborn. Rasterio, Rioxarray.
    \end{indentpar}
    
    \begin{indentpar}{0.5cm}
    GitHub : https://github.com/yclipse
    \end{indentpar}
    
\end{rSection}

%====================================================
\begin{rSection}{Education}
{\bf University of Edinburgh, UK} 

\begin{indentpar}{0.5cm}
    \textbf{PhD in Petrophysics} \hfill {Oct 2016 - Apr 2021}
    
    \textbf{MSc(Research) in Geology, Distinction} \hfill {August 2015 - August 2016} 
    
    \textbf{BSc Geology, 2.1} Double-degree programme \hfill {Sep 2011 - May 2015} 
\end{indentpar}

\end{rSection}
%====================================================

\begin{rSection}{ACADEMIC EXPERIENCES}
\textbf{International Centre for Carbonate Reservoirs (ICCR), UoE with Petrobras and Shell} \hfill {2016-2020}
\end{rSection}

%===============================================

\begin{rSection}{Personal Side Project}
Writing scientific outreach articles at zhihu.com

    \begin{indentpar}{0.5cm}
    41k Followers, 178k Likes, 22k Thanks, 71k Stars, 4 Editor's recommendation
    \end{indentpar}

\end{rSection}
%=====================================================
\end{document}
