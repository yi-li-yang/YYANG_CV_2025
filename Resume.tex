%%%%%%%%%%%%%%%%%%%%%%%%%%%%%%%%%%%%%%%%%
% Medium Length Professional CV
% LaTeX Template
% Version 2.0 (8/5/13)
%
% This template has been downloaded from:
% http://www.LaTeXTemplates.com
%
% Original author:
% Rishi Shah 
%
% Important note:
% This template requires the resume.cls file to be in the same directory as the
% .tex file. The resume.cls file provides the resume style used for structuring the
% document.
%
%%%%%%%%%%%%%%%%%%%%%%%%%%%%%%%%%%%%%%%%%

%----------------------------------------------------------------------------------------
%	PACKAGES AND OTHER DOCUMENT CONFIGURATIONS
%----------------------------------------------------------------------------------------
\renewcommand{\baselinestretch}{0.70}
\documentclass{resume} % Use the custom resume.cls style
\newenvironment{indentpar}[1]%
  {\begin{list}{}%
          {\setlength{\leftmargin}{#1}}%
          \item[]%
  }
  {\end{list}}

\usepackage[left=0.5in,top=0.3in,right=0.5in,bottom=0.4in]{geometry} % Document margins
\newcommand{\tab}[1]{\hspace{.2\textwidth}\rlap{#1}}
\newcommand{\itab}[1]{\hspace{0em}\rlap{#1}}
\name{YILI YANG} % Your name
\address{(+44)7490334233\\ yyl.eli@gmail.com; \\
Room 400 the Grant Institute, James Hutton Road, Edinburgh EH9 3FE} % Your phone number and email
\begin{document}

%=================================================================================================================================================
\begin{rSection}{Education}
{\bf University of Edinburgh, UK} 

\textbf{    PhD Candidate in Geology} \hfill {Expected 06/2019}
\begin{indentpar}{0.5cm}
\textit{Multiphase Fluid Flow and Trapping in Porous Carbonates: A synchrotron X-ray CT and pore-scale modelling investigation. (Fully funded by Petrobras and Shell)}
\end{indentpar}
\textbf{    MSc(Research) in Geology}, Distinction \hfill {August 2015 - August 2016} 
\begin{indentpar}{0.5cm}
\textit {Processing and analysis of time-resolved X-ray μCT data of experimental dolomitisation}
\end{indentpar}
\textbf{    BSc Geology}, 2.1 (2+2 double-degree programme with CDUT) \hfill {Sep 2011 - May 2015} 
\end{rSection}
%=================================================================================================================================================
\begin{rSection}{Technical Strengths}

\begin{indentpar}{0.5cm}
\textbf{Programming Language} Python ($4^+$year experience) with good knowledge of science and data related libraries such as Scipy, Pandas, Pytorch, Tomopy etc. 

\textbf{Professional Software} Avizo, ImageJ, Octopus, Dragonfly, ParaView

\textbf{Data Processing and Analysis}  Experienced in image processing including reconstruction, denoising, segmentation and visualisation, deep learning segmentation and statistical analysis of scientific data

\textbf{Scientific Experiments}  Core-flooding experiments, conventional and synchrotron CT imaging 

\textbf{Language}  Proficient English, native Chinese
\end{indentpar}
\end{rSection}
\begin{rSection}{Work Experience}
\begin{rSubsection}{Teaching Assistant - University of Edinburgh}{June 2016-2019}{}

\begin{indentpar}{0.5cm}
\textbf{Field Demonstrator} Demonstrated in the undergraduate geological field trips in Inchnadamph (3), Kinlochleven (1) and Cyprus (1). Teaching field observation and mapping techniques. Explaining basic geological concepts and principles. Answering questions about field observations and course works. Collaborating with other demonstrators to cover logistics, health and safety issues during field trips. Trained and certificated for emergency first aid skills in case of emergency.

\textbf{Course Demonstrator} Demonstrated in the undergraduate course Geochemical Data Processing and Analysis. Teaching basic skills of scientific data analysis using python including: data processing, quantified, statistical analysis, and scientific data visualisation. Helping students to use fundamental python packages such as Scikit and Matplotlib. Helping students with course works and tests. Collaborating with other demonstrators to organise workshops.
\end{indentpar}

\end{rSubsection}
\begin{rSubsection}{Intern - Brighter Oil Group, Xian, China}{2015 Summer Intern}{}

\begin{indentpar}{0.5cm}
Project: Evaluation of velocity string on enhanced hydrocarbon recovery in Sulige gas field, Inner Mongolia, China. Research and presentation focusing on the application feasibility of velocity string as coiled tubing. Analysing the cost-effectiveness of velocity string in Sulige gas field. Studying and comparing the successful case of velocity string application on enhanced hydrocarbon recovery.
\end{indentpar}

\end{rSubsection}
\end{rSection}
%=================================================================================================================================================
\begin{rSection}{ACADEMIC EXPERIENCES}
\textbf{School of Geosciences, The University of Edinburgh} 
\begin{indentpar}{0.5cm}
\textit{Multiphase Fluid Flow and Trapping in Porous Carbonates (Thesis study)}
\begin{itemize}
    \item Core-flooding experiments on benchmark carbonate imaged with X-ray synchrotron microtomography
    \item Developed a high quality CT image processing work flow with parallel computation enabled.
    \item Developed an intergrated python-based CT image processing toolkit 
    \item Trained a convolutional neural network model for CT image segmentation
    \item Identification, visualisation, quantification and analysis of various of pore-scale displacement events.
\end{itemize}
\end{indentpar}

\textbf{International Centre for Carbonate Reservoirs (ICCR), Petrobras and Shell} 
\begin{indentpar}{0.5cm}
\textit{Experimental and modelling study of the Pre-salt hydrocarbon reservoir, Brazil, from pore to reservoir-scale}
\begin{itemize}
    \item Providing experimental data and observations to the modelling groups 
    \item Providing image processing support to other experimental groups
    \item Experimental validation of numerical models on multiphase flow in porous media
\end{itemize}
\end{indentpar}
\textbf{Experiments} 

\begin{indentpar}{0.5cm}
\textit{Gypsum dehydration using synchrotron X-ray imaging, PSICHE, Switzerland} \hfill {2019}

\textit{Core-flooding experiments on reservoir rocks using in-house CT imaging, Edinburgh} \hfill {2018}

\textit{Core-flooding experiments on Indiana limestone using synchrotron X-ray imaging, APS, ANL, U.S.} \hfill {2017}
\end{indentpar}

\textbf{Training} Open Pit Mining for the Extraction of Solid Mineral Resources \hfill {2015}

\textbf{Geological Field works}
\begin{indentpar}{0.5cm}
\textbf{Cyprus} Oceanic crust and ophiolite complex\hfill{2015}

\textbf{Inchnadamph, Scotland} Thrust nappes and tectonic window \hfill{2014}

\textbf{Alicante, Spain} Sedimentary basins and carbonate sediments\hfill{2014}

\textbf{Kinlochleven, Scotland} Rock deformations and structural geology\hfill{2014}

\textbf{Helmsdale, Scotland} Petroleum system of the North Sea\hfill{2013}

\textbf{Majiaoba, China} Foreland Basin and imbricate thrust stacks\hfill{2013}

\textbf{Emeishan, China} Emeishan basalt, P-T boundary and sedimentary units\hfill{2012}

\end{indentpar}
\end{rSection}
%=================================================================================================================================================
\begin{rSection}{Publications and Conferences}
\textbf{Publications}
\begin{indentpar}{0.5cm}
\begin{itemize}
    \item An implementation of a convolutional neural network for fast segmentation of 4D microtomography volumes from fluid flow experiments in porous media (Yang et.al. 2019, in prep.)
    \item Numerical investigation of a Roof snap-off observed in an Indiana limestone core flood experiment (Maes and Yang, 2019, in prep.)
    \item Chemical-mechanical-hydraulic coupling in deforming, dehydrating halite-gypsum rocks - implications for basal detachments in thin-skinned tectonics (Sina Marti et al. 2019 in prep.)
\end{itemize}
\end{indentpar}

\textbf{Conferences}
\begin{indentpar}{0.5cm}
\textbf{Interpore}, Valencia, Spain \hfill {2019}
\begin{indentpar}{0.5cm}
\textit {Speak: "Fluid displacement and trapping during two-phase steady-state flow in complex carbonate imaged by synchrotron x-ray microtomography"}
\end{indentpar}

\textbf{AGU Fall}, Washinton DC, U.S. \hfill {2018}
\begin{indentpar}{0.5cm}
\textit {Speak: "Fluid Connectivity Evolution during Drainage-Imbibition Cycle in Porous Carbonate Rock"}
\end{indentpar}

\textbf{ICCR Conference} ICCR Annual Conference with Shell and Petrobras, RJ, Brazil \hfill {2018}
\begin{indentpar}{0.5cm}
\textit {Speak: "SatuTrack II - Experimental Multi-phase Fluid Displacement in Presalt Reservoir Rocks"}

\textit {Workshop: "Image processing and segmentation for X-ray microtomography"} Universidade Federal do Rio de Janeiro, Brazil
\end{indentpar}

\textbf{ICCR Conference} ICCR Annual Conference with Shell and Petrobras, RJ, Brazil \hfill {2017}
\begin{indentpar}{0.5cm}
\textit {Speak: "SatuTrack II - Saturation tracking and identification of residual oil distributions"}
\end{indentpar}

\textbf{Postgraduate Research Conference} The University of Edinburgh \hfill {2017} 
\begin{indentpar}{0.5cm}
\textit {Speak: "Multiphase Fluid Flow and Trapping in Porous Carbonates"} (Presentation Prize Winner)
\end{indentpar}
\end{indentpar}
\end{rSection}
%=================================================================================================================================================

\end{document}
