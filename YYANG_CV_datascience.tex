%%%%%%%%%%%%%%%%%%%%%%%%%%%%%%%%%%%%%%%%%
% Medium Length Professional CV
% LaTeX Template
% Version 2.0 (8/5/13)
%
% This template has been downloaded from:
% http://www.LaTeXTemplates.com
%
% Original author:
% Rishi Shah 
%
% Important note:
% This template requires the resume.cls file to be in the same directory as the
% .tex file. The resume.cls file provides the resume style used for structuring the
% document.
%
%%%%%%%%%%%%%%%%%%%%%%%%%%%%%%%%%%%%%%%%%

%----------------------------------------------------------------------------------------
%	PACKAGES AND OTHER DOCUMENT CONFIGURATIONS
%----------------------------------------------------------------------------------------
\renewcommand{\baselinestretch}{0.70}
\documentclass{resume} % Use the custom resume.cls style
\newenvironment{indentpar}[1]%
  {\begin{list}{}%
          {\setlength{\leftmargin}{#1}}%
          \item[]%
  }
  {\end{list}}

\usepackage[T1]{fontenc}
\usepackage[utf8]{inputenc}
\usepackage{mathptmx}

\usepackage[left=0.5in,top=0.35in,right=0.5in,bottom=0.4in]{geometry} % Document margins
\newcommand{\tab}[1]{\hspace{.25\textwidth}\rlap{#1}}
\newcommand{\itab}[1]{\hspace{0em}\rlap{#1}}
\name{YILI YANG} % Your name
\address{+44 7490334233   \textbar yyl.eli@gmail.com  \textbar
 28/3 Montrose Terrace, Edinburgh, UK EH7 5DL} % Your phone number and email



\begin{document}
%-------------------------------------------------------------
\begin{rSection}{Work Experience}
{\bf Woodwell Climate Research Center, MA, US (remote)} \hfill {Jan 2022 - } 

\begin{indentpar} {0.5cm} Data Scientist
\begin{indentpar} {0.5cm} Project lead
\end{indentpar}
\begin{indentpar} {0.5cm} Semantic segmentation of retrogressive thaw slumps using deep learning. Collaborated with parallel projects such as wildfire detection and waterbody detection. Developed a python package for satellite imagery processing and Earth Engine data processing.
\end{indentpar}

\end{indentpar}
{\bf Faculty.ai, London, UK} \hfill {Oct 2021 - Dec 2021} 
\begin{indentpar}{0.5cm}

Data Science Fellowship Programme
\begin{indentpar}{0.5cm}
Intensive hands-on training in data science and commercial skills followed by an industry placement


{\bf Industry placement: Woodwell Climate Research Center, MA, US}

Semantic segmentation of permafrost thaw slump landscape on satellite imagery data using deep neural networks. Used Google Earth Engine to acquire and process data. Used image augmentation techniques to enlarge the dataset. Used Tensorflow and Keras to train a U-net model. The model performance reached the highest score of a publication in 2021 on the same task. New potential thaw slump locations were discovered by the model.
\end{indentpar}
\end{indentpar}
\end{rSection}
%===============================================================
\begin{rSection}{Technical Strengths}

\begin{indentpar}{0.5cm}
\textbf{Coding Languages} Python (7-year experience): Numpy, Pandas, Sklearn, Skimage, PyTorch, TensorFlow/Keras, OpenCV and other libraries. Markdown, Command-line and LaTeX. Basic SQL and JavaScript.

\textbf{Machine Learning} Semantic segmentation of synchrotron microtomography images and satellite imagery using deep neural networks. Modelling carbon emission with tree-based models.

\textbf{Data Processing} Structured/unstructured data cleaning, processing, analysis and visualisation. Image data processing, denoising, augmentation and segmentation. Geo-spatial data processing. Data processing workflow.

\textbf{Tools} Jupyter Notebook, Google Colab, Git, Google Earth Engine, Google Cloud Platform

\end{indentpar}
\end{rSection}

%====================================================
\begin{rSection}{Education}
{\bf University of Edinburgh, UK} 
\begin{indentpar}{0.5cm}

\textbf{PhD in Petrophysics} \hfill {Oct 2016 - Apr 2021}
\begin{indentpar}{0.5cm}
Project: \textit{Multiphase Fluid Flow and Trapping in Porous Carbonates: A synchrotron X-ray CT and pore-scale modelling investigation. (Fully funded by Petrobras and Shell)}
\end{indentpar}
\textbf{MSc(Research) in Geology, Distinction} \hfill {August 2015 - August 2016} 
\begin{indentpar}{0.5cm}
Project: \textit {Processing and analysis of time-resolved X-ray μCT data of experimental dolomitisation}
\end{indentpar}
\textbf{    BSc Geology, 2.1}  \hfill {Sep 2011 - May 2015} 
\end{indentpar}
\end{rSection}
%====================================================
\begin{rSection}{DEVELOPMENT EXPERIENCE}
\begin{indentpar}{0.5cm}
\textbf{International Centre for Carbonate Reservoirs (ICCR), Petrobras and Shell} 
\begin{indentpar}{0.5cm}
\begin{itemize}   
    \item Developed a high-quality CT image processing workflow 
    \item Developed an integrated python-based CT image processing toolkit 
    \item Developed algorithm for measuring the representative elementary volume of porous media.
\end{itemize}
\end{indentpar}
\end{indentpar}
\end{rSection}
%===============================================

%=====================================================
\end{document}
