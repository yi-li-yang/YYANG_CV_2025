%%%%%%%%%%%%%%%%%%%%%%%%%%%%%%%%%%%%%%%%%
% Medium Length Professional CV
% LaTeX Template
% Version 2.0 (8/5/13)
%
% This template has been downloaded from:
% http://www.LaTeXTemplates.com
%
% Original author:
% Rishi Shah 
%
% Important note:
% This template requires the resume.cls file to be in the same directory as the
% .tex file. The resume.cls file provides the resume style used for structuring the
% document.
%
%%%%%%%%%%%%%%%%%%%%%%%%%%%%%%%%%%%%%%%%%

%----------------------------------------------------------------------------------------
%	PACKAGES AND OTHER DOCUMENT CONFIGURATIONS
%----------------------------------------------------------------------------------------
\renewcommand{\baselinestretch}{0.99}
\documentclass{resume} % Use the custom resume.cls style
\newenvironment{indentpar}[1]%
  {\begin{list}{}%
          {\setlength{\leftmargin}{#1}}%
          \item[]%
  }
  {\end{list}}

\usepackage[T1]{fontenc}
\usepackage[utf8]{inputenc}
\usepackage{mathptmx}
\linespread{1}
\fontsize{12}{13.5}

\usepackage[left=1in,top=1in,right=1in,bottom=1in]{geometry} % Document margins
\newcommand{\tab}[1]{\hspace{.25\textwidth}\rlap{#1}}
\newcommand{\itab}[1]{\hspace{0em}\rlap{#1}}
\name{YILI YANG} % Your name
\address{+44 7490334233\textbar\textbar yyang@woodwellclimate.org\textbar\textbar ORCID 0000-0002-1791-3899} % Your phone number and email

\begin{document}
%-------------------------------------------------------------
%====================================================
\begin{rSection}{Research Interest}
My research focuses on developing advanced AI methodologies to address complex scientific problems across disciplines, with particular expertise in applying deep learning to geoscience, environmental and climate solutions. At the interface of computer science and earth systems, I specialise in multimodal data fusion, time-series modelling, geospatial regression, semantic image segmentation, multivariate regression problems. I work with multimodal data including satellite imagery, field observations, sensor data, and process model-derived data. Through interdisciplinary collaborations spanning geoscience, environmental science and ecology, I aim to create innovative approaches that accelerate scientific discovery and prediction, while developing hybrid approaches that combine AI-driven analytics with traditional science to transform how we understand and respond to our changing planet.

\end{rSection}
%====================================================
\begin{rSection}{Education}
{\bf University of Edinburgh, UK} 

\begin{indentpar}{0.5cm}
\textbf{PhD in Petrophysics} \hfill {Oct 2016 - Apr 2021}

\textbf{MSc (Research) in Geology, Distinction} \hfill {August 2015 - August 2016}

\textbf{BSc Geology, Upper Second} \hfill {Sep 2011 - May 2015} 
\end{indentpar}

\end{rSection}
%====================================================
\begin{rSection}{Experience}
{\bf Data Scientist, Woodwell Climate Research Center, MA, USA (remote)} \hfill {Jan 2022 - } 

\begin{itemize}
    \item Project Lead: the pan-Arctic Retrogressive Thaw Slumps (RTS) mapping initiative, applying state-of-the-art deep learning techniques to satellite imagery for detecting and quantifying permafrost thaw features, directly contributing to climate feedback modeling and carbon cycle research
    \item Designed and built the ARTS scientific dataset, establishing a benchmark training resource for deep learning applications in environmental monitoring
    \item Developed innovative machine learning workflows for multimodal data integration, combining various satellite imagery types and field observations to enhance feature detection accuracy
    \item Collaborated in two wild fire projects using deep learning for wild fire mapping on satellite imagery and using environmental variables to predict wild fire using machine learning
    \item Collaborated in carbon flux time series imputation using machine learning
    \item Lead annual machine learning workshop series for Woodwell researchers, providing training in advanced AI techniques for environmental science applications including image processing and time-series forecasting. Designing progressive curriculum materials, adapting teaching methods for diverse learner backgrounds.
    \item Mentored annual summer interns, created training materials for interns and research assistants. Works were presented on the AGU annual conference.
\end{itemize}

{\bf Data Science Fellow, Faculty.ai, London, UK} \hfill {Sep 2021 - Dec 2021}
\begin{itemize}
    \item Completed intensive fellowship in Machine Learning and Artificial Intelligence, focusing on developing practical AI solutions for complex data challenges. Include trainings and courses for professional data scientist and an industry placement.
\end{itemize}

{\bf PhD Researcher, International Centre for Carbonate Reservoirs, Edinburgh, UK}
\hfill{Sep 2017 - Sep 2020}
\begin{itemize}
    \item Conducted doctoral research on multiphase fluid flow in porous media, developing novel computational methods for analyzing complex flow dynamics, funded by Petróleo Brasileiro
    \item Parcipitated in petrophysical experiments at the Diamond Light Source UK, the Advanced Photon Source Argonne National Laboratory, US and the Swiss Light Source Paul Scherrer Institute, CH
\end{itemize}
\end{rSection}
%===============================================================
\begin{rSection}{SELECTED PUBLICATIONS}

\underline{Yang Y}, Rogers B M, Fiske G, et al. Mapping retrogressive thaw slumps using deep neural networks. Remote Sensing of Environment[J], 2023

\underline{Yang Y}, Rodenhizer H, Rogers B M, et al. A Collaborative and Scalable Geospatial Data Set for Arctic Retrogressive Thaw Slumps with Data Standards[J]. Scientific Data, 2025, 12(1): 18.

Rodenhizer H, \underline{Yang Y}, Fiske G, et al. A Comparison of Satellite Imagery Sources for Automated Detection of Retrogressive Thaw Slumps[J]. Remote Sensing, 2024, 16(13): 2361.

Mullen A, Watts J D, Rogers B M, ... \underline{Yang Y} et al. Using High-resolution Satellite Imagery and Deep Learning to Track Dynamic Seasonality in Small Water Bodies. Geophysical Research Letters[J], 2023

Li W, Hsu C Y, Wang S, ... \underline{Yang Y} et al. Segment Anything Model Can Not Segment Anything: Assessing AI Foundation Model’s Generalizability in Permafrost Mapping[J]. Remote Sensing, 2024, 16(5): 797.

Potter S, \underline{Yang Y}, Burrell A, et al. Mapping Boreal and Tundra North American Burned Area using Convolutional Neural Networks". the International Journal of Wildland Fire[J], Submitted.

Gu Z, Li W, Hsu C, \underline{Y Yang}, et al. A Multi-Scale Vision Transformer-Based Multimodal Geoai Model for Mapping Arctic Permafrost Thaw[J]. Available at SSRN 4762408.


\end{rSection}
%=====================================================
\begin{rSection}{DATASETS}

\underline{Yili Yang}, Heidi Rodenhizer, Jacqueline Dean. (2024). Arctic Retrogressive Thaw Slumps (ARTS): digitisations of pan-Arctic retrogressive thaw slumps, 1985-2021. Arctic Data Center. urn:uuid:cbeb4511-55ca-4f9a-b7e5-4154184d96d0.

\end{rSection}
%=====================================================
\begin{rSection}{RESEARCH FUNDINGS}

Funding for Climate Solutions: A Generic Climate AI Framework for Multi-domain Time Series Prediction, Woodwell Climate Research Center, \$99,749
    
Overall Objectives: This project will attempt to develop a generic climate AI framework that will unify and accelerate time-series inferences across Woodwell Climate Research Center's diverse research domains, from Arctic carbon fluxes to tropical forest dynamics.

Pending/ongoing funding applications with NASA, NSF, Google.org and the Bezos Foundation
    
\end{rSection}
%=====================================================
\begin{rSection}{PARTICIPATED RESEARCH INITIATIVES}

\textbf{Permafrost Discovery Gateways} Funded by Google.org and NSF, the Gateway is making information of permafrost conditions available throughout the Arctic by providing access to big geospatial products and tools to allow exploration and discovery for researchers, educators, and the public at large. This includes providing automated monthly monitoring of permafrost thaw during the snow-free season from satellite imagery.

\textbf{Permafrost Pathways} Funded through the TED Audacious Project — a collaborative funding initiative catalyzing big, bold solutions to the world’s most urgent challenges. Through a joint effort between Woodwell Climate Research Center, the Arctic Initiative at Harvard Kennedy School, and the Alaska Institute for Justice, Permafrost Pathways brings together leading experts in climate science, policy action, and environmental justice to inform and develop adaptation and mitigation strategies to address permafrost thaw.


\end{rSection}
%=====================================================
\begin{rSection}{ACADEMIC SERVICES, PRESENTATIONS AND TALKS}
\begin{itemize}
    \item Peer-reviewer for five manuscripts \hfill 2022-2024
    \item Invited webinar at the Royal Meteorological Society: Machine Learning for Atmospheric Sciences: Values and Controversies, UK \hfill 2022
    \item Invited Panel talk at the Esri UC Spatial Analytics Summit, US \hfill 2024
    \item Invited talk at the Cyber2A workshop, US \hfill 2024
    \item European Geophysical Union annual conference, EU \hfill 2022, 2023, 2024
    \item Google Geo-for-Good Conference, US \hfill 2022, 2023
    \item Tri-polar remote sensing conference, China \hfill 2023, 2024
    \item American Geophysical Union annual conference, US \hfill 2018
    \item RTSInTrain workshop, EU \hfill 2023
    \item NASA Arctic-Boreal Vulnerability Experiment workshop, US \hfill 2022
    \item Permafrost Discovery Gateway webinar \hfill 2022
\end{itemize}


\end{rSection}
%=====================================================
\end{document}
